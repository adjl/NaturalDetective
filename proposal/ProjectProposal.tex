\documentclass[a4paper]{article}

\usepackage[UKenglish]{babel}
\usepackage[backend=biber]{biblatex}
\usepackage[T1]{fontenc}
\usepackage[utf8]{inputenc}
\usepackage{color}
\usepackage{csquotes}
\usepackage{lmodern}

% Custom commands ===============================================================
\newcommand{\comment}[1]{[\textcolor{red}{#1}]} % \comment: Write comments in red
% ===============================================================================

\title{Natural Detective: A Serious Game for Learning \\ Natural Deduction in Propositional Logic \\
  \medskip \Large{Project Proposal}}
\author{Helena Josol}
\date{}

\addbibresource{References.bib}

\begin{document}

\maketitle


\section{Problem Description}

\subsection{Background}
Natural deduction, along with other higher concepts in mathematics, is often first introduced and taught at the university level. Under the instruction of a professor and tutors, a student's learning on the subject is directed and focused on its essential parts. This ensures that the student will have familiarity and understanding of the main ideas, as opposed to a broad yet shallow overview of the entire topic.

But without instructors, educating oneself on this proof calculus---as with other advanced topics, mathematics or not---can be a challenge given limited knowledge. Lacking guidance and prior experience, a self-taught student will be unable to correctly define the scope of their learning, identify integral points from unnecessary details, stretch their understanding further, properly evaluate their mistakes, nor keep themselves wholly engaged and accountable to their studies.

\comment{I need to talk more about natural deduction, rather than just mentioning it. The ``subject'' of ND doesn't flow very well---it somehow needs to bridge.}

What is found wanting then, in the case of self-education, is the presence of an active agent of teaching, i.e., merely having the resources (e.g.\ books, notes) available does not guarantee full absorption of the material. A human educator has always filled this void in more traditional settings, and nowadays an equivalent technological solution is fortunately becoming more evident and accepted: serious games.

\emph{Serious games} are an emerging technology which employ the principles of gamification for pedagogical value. They provide entertaining yet effective ways to teach educational subjects, and act as accessible platforms to conduct training for corporate or military purposes. When not packaged as such, these topics and skills are normally taught in regular fashion (e.g.\ lecture, simulation), some of which render the material tedious, dull, or even difficult to learn.

Because serious games add an active element to the learning process, in contrast to the passive mode of students self-teaching ``dry'' academic subjects, there are benefits to be gained in placing a serious game in the role of a ``teacher'', especially in a self-educating context.

\comment{Citations needed for this section.}

\subsection{Advantages of Serious Games}
As mentioned previously, serious games can address issues with, and can bring much benefit to, the area of self-learning. The advantages to utilising this educational medium are as follows:
\begin{enumerate}
  \item What will I do?
    \begin{itemize}
      \item \draft{To build an educational game that teaches introductory natural deduction in propositional logic.}
    \end{itemize}
    \begin{enumerate}
      \item Why is it meaningful / relevant?
      \item Why is it interesting?
      \item An outstanding problem
      \item Aims: \textit{\dots identify, at a high level, what you hope to achieve}
      \item Objectives: \textit{\dots quantitative and qualitative measures by which the completion of your project will be judged (How to know that my project is a success? Setting my own assessment criteria.)}
        \begin{itemize}
          \item Objective 1: Design and build a system that does\dots
          \item Objective 2: Evaluate its performance in terms of\dots
          \item Objective 3: Present guidelines for future systems that do\dots
        \end{itemize}
    \end{enumerate}
  \item Why will I do it?
    \begin{itemize}
      \item \draft{Games are good for education and are very suitable to make mathematical concepts more accessible. Features:}
      \begin{itemize}
        \item \draft{Accessibility: games are easy to pick up (unlike, say, a textbook). One reason is that they can restrain the things you can do: they will not allow you to make "illegal" moves, and they can limit the available moves to those relevant to the problem. In contrast, with exercises in a textbook nobody tells you how to get started, and whether what you're doing makes sense.}
        \item \draft{Engagement: similarly, games are hard to put down. That is because they're fun (at least, that's the intention). The consequence is that you can spend a long time learning something without even noticing.}
        \item \draft{Instant feedback: a game will tell you immediately when your answer is wrong. Every theory of learning will tell you that that's very important, since the sequence of thoughts leading you to a wrong answer should still be fresh, if you want to find out where you went wrong. This is why tutorials are so good, compared to (most) lectures, and why good tutors are essential.}
      \end{itemize}
    \end{itemize}
  \item \textbf{Self-contained environment.} A student will not need to check and cross-reference among many resources to fill in gaps in the material, as is the case when perusing, say, textbooks aimed at varying levels of expertise, which themselves may use different symbols and terminology to further add confusion. Having a learning environment designed as a coherent whole will lessen the need for a thorough background in the field, which then leaves it free to emphasise only what is important.
  \item \textbf{Low barrier to entry.} Compared to most---if not, all---learning mediums, a game is welcoming and easier to begin using. As is characteristic of most games, a serious game can provide its own set of tutorials on how to manipulate the system. But the beauty comes in when the tutorials already teach the fundamentals of the subject subtly and gradually---the student begins to learn before they even realise!
  \item \textbf{Interactivity.} A game enables, not just one, but a two-way continuous interaction between itself and the player. In the context of teaching, this is akin to a personal tutor taking careful note of the student's every action (e.g.\ a question asked, a step in a proof), which is beneficial in keeping the student's understanding correct, and their work sensible. Of course, a tutor is never always present, nor is one ever there when self-studying. Serious games address this lack.
  \item \textbf{Ease-of-use (constraints).} Closely related to (3) is the ability of the game to constrain ``moves'' in response to player actions. For example, the number of possible succeeding moves is decreased given the previous one, or a certain action is denied because it is illegal. Some form of this assistance can be given by a tutor (only when they notice), but is simply absent when studying on one's own.
  \item \textbf{Active feedback.} Again, similar to (3), a game has the capability of providing instantaneous and aggregate feedback. If a mistake were to be made, then the system can immediately highlight the error, allowing the player to correct and identify what they misunderstood. Additionally, assessments can be made on overall performance, citing areas of possible improvement (e.g.\ using more steps than necessary). Although mentioned and listed under (4), disallowing invalid moves, and allowing a subset of legal ones are themselves forms of feedback.
  \item \textbf{Adapting difficulty.} To great benefit, game feedback behaviour as described in (4) and (5) can be adjusted to suit the player's individual needs, i.e., the enforced \emph{helpfulness} and \emph{strictness} of the game. As an example, a student who wishes for a more challenging environment can disable warnings and filtering of valid moves. Similarly, a student who would like to check their understanding step-wise is also able to do so.
  \item \textbf{Non-traditional (creative) presentation.} \comment{TBC}
  \item \textbf{Experimentation and playfulness.} \comment{TBC} % One feature that theSimilarly, a player can pick and choose various options and see what can be valid.
  \item \textbf{Engagement and fun.} \comment{TBC}
\end{enumerate}

\section{Scope \& Limitations} % NB: Refer to the slides as you write this
Why didn't you\dots? Why did you focus on\dots?
% \item Engagement: similarly, games are hard to put down. That is because they're fun (at least, that's the intention). The consequence is that you can spend a long time learning something without even noticing.
% depending on how they deliver the subject, can help make it interesting, entertaining, and (at best) fun.

\section{Requirements} % NB: Refer to the slides as you write this
Functional and non-functional requirements, specification (table)?
In summary, serious games are excellent educational tools, and are very good candidate media for presenting mathematical concepts---especially the advanced and ``dull'' ones---in a more accessible and friendly manner.

\section{Game Design} % NB: Refer to the slides as you write this
How will I do it?: Have a rudimentary game idea already. Ask Willem for feedback and input.
\comment{Citations sorely needed for this section.}

\subsection{Aim}
The aim of the project is to design and build a serious game that teaches \emph{natural deduction in propositional logic}, then evaluating its effectiveness in helping players learn. \comment{Not happy with this; just like with 1.1, I need to state why I chose natural deduction above other topics.}

\subsection{Learning Objective}
What do you want the player to learn? (e.g.\ Learn to construct a finite automaton)

\subsection{Player Goals}
e.g.\ Construct a finite automaton corresponding to a given regular expression

\subsection{Player Moves}
e.g.\ Add states; add transitions; make final states; change transition labels; validate answer

\subsection{Interface}
The challenge is in making a good interface.

\subsection{Feedback \& Rewards}

\subsection{Level Design}
Collaborate with Willem on this.

\subsection{Platform}


\section{Project Management} % NB: Refer to the slides as you write this
\subsection{Gantt Chart}
\subsection{Waypoints and Endpoints}

\section{Resources} % NB: Refer to the slides as you write this
Consider resources and their availability

\newpage
\printbibliography

\end{document}